\begin{abstract}

%  Big Topic Intro  
Machine learning models---deep neural networks in particular---have performed remarkably well on benchmark datasets across a wide variety of domains. However, the ease of finding adversarial counter-examples remains a persistent problem when training times are measured in hours or days and the time needed to find a successful adversarial counter-example is measured in seconds.
%  Problem Statement and Drawbacks  
Much work has gone into generating and defending against these adversarial counter-examples, however the relative costs of attacks and defences are rarely discussed. Additionally, machine learning research is almost entirely guided by test/train metrics, but these would require billions of samples to meet industry standards. The present work addresses the problem of understanding and predicting how particular model hyper-parameters influence the performance of a model in the presence of an adversary.
%  Our solution and Advantages of our solution  
The proposed approach uses survival models, worst-case examples, and a cost-aware analysis to precisely and accurately reject a particular model change during routine model training procedures rather than relying on real-world deployment, expensive formal verification methods, or accurate simulations of very complicated systems (\textit{e.g.}, digitally recreating every part of a car or a plane).
%  Conclusion  TODO quantify or examplify this somehow?
Through an evaluation of many pre-processing techniques, adversarial counter-examples, and neural network configurations, the conclusion is that deeper models do offer marginal gains in survival times compared to more shallow counterparts. However, we show that those gains are driven more by the model inference time than inherent robustness properties. Using the proposed methodology, we show that ResNet is hopelessly insecure against even the simplest of white box attacks.

\end{abstract}